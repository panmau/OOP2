%import template
\documentclass[a4paper, landscape , 8pt]{scrartcl}

% use language german
\usepackage[T1]{fontenc}
\usepackage[utf8]{inputenc}
\usepackage[english, ngerman]{babel} % \selectlanguage{english} if  needed
\usepackage{lmodern} % use modern latin fonts

% format
\usepackage{geometry}
\geometry{top=1.2cm,left=0.4cm,right=0.4cm}
\textheight = 558pt

%autor
\usepackage{authblk}

%tabular
\usepackage{tabularx}

% math
\usepackage{amsmath}
\usepackage{amssymb}
\usepackage{amsfonts}
\usepackage{enumitem}

% graphic
\usepackage{graphicx}
\graphicspath{{graphic/}} 

%colors
% \usepackage{xcolor}

% Multi Columns
\usepackage{multicol}

% more compact items. explicit parameters as I want some separation between items
% see https://tex.stackexchange.com/questions/10684/vertical-space-in-lists
\setlist{topsep=0pt, leftmargin=4mm,itemsep=1.5pt,partopsep=0pt, parsep=0pt}
\setlength{\parindent}{0cm}

%define header and footer
\usepackage{fancyhdr}
\pagestyle{fancy}

\fancyhead[RO]{\AUTHOR| \INSTITUTE}
\fancyhead[LO]{\TITLE}
\fancyfoot[RO]{09.01.2022}
\renewcommand\headrulewidth{0pt}
\renewcommand\footrulewidth{0pt}
\headsep = -2pt
\footskip = 0pt


% Define Section Format
\usepackage{sectsty}
\usepackage{titlesec}
\usepackage[dvipsnames]{xcolor}

\titleformat{name=\section}[block]{\sffamily\normalsize}{}{0pt}{\colorsection}
\titlespacing*{\section}{0pt}{0pt}{0pt}
\newcommand{\colorsection}[1]{%
	\colorbox{sectioncolor!80}{\parbox{0.98\linewidth}{\vspace{-1pt}\color{white}\ #1 \vspace{-2pt}}}}

% Define Subsection Format
\titleformat{name=\subsection}[block]{\sffamily\small}{}{0pt}{\colorsubsection}
\titlespacing*{\subsection}{0pt}{0pt}{0pt}
\newcommand{\colorsubsection}[1]{%
	\colorbox{subsectioncolor!80}{\parbox{0.98\linewidth}{\vspace{-1pt}\color{black}\ #1 \vspace{-2pt}}}}

% Define SubSubsection Format
\titleformat{name=\subsubsection}[block]{\sffamily\small}{}{0pt}{\colorsubsubsection}
\titlespacing*{\subsubsection}{0pt}{0pt}{0pt}
\newcommand{\colorsubsubsection}[1]{%
	\colorbox{subsubsectioncolor!60}{\parbox{0.98\linewidth}{\vspace{-1pt}\color{black}\ #1 \vspace{-2pt}}}}


%define color
\definecolor{sectioncolor}{HTML}{052639}
\definecolor{subsectioncolor}{HTML}{468189}
\definecolor{subsubsectioncolor}{HTML}{8DB9B1}
\definecolor{b}{RGB}{0, 115, 192 } %Default highlight color
\definecolor{p}{RGB}{0, 43, 54} %Dark page color
\definecolor{t}{RGB}{131, 148, 150} %Dark text color
\definecolor{darkgreen}{RGB}{0,150,0}
\definecolor{dkgreen}{rgb}{0,0.6,0}
\definecolor{gray}{rgb}{0.5,0.5,0.5}
\definecolor{mauve}{rgb}{0.58,0,0.82}
\definecolor{DarkPurple}{rgb}{0.4, 0.1, 0.4}
\definecolor{DarkCyan}{rgb}{0.0, 0.5, 0.4}
\definecolor{LightLime}{rgb}{0.3, 0.5, 0.4}
\definecolor{Blue}{rgb}{0.0, 0.0, 1.0}
\definecolor{h}{RGB}{1, 101, 163}

% Code Listings
\usepackage{listings}
\usepackage{color}
\usepackage{beramono}
\usepackage{hyperref}
\hypersetup{
    colorlinks,
    linkcolor={black},
    citecolor={blue!50!black},
    urlcolor={blue!80!black}
}

\definecolor{bluekeywords}{rgb}{0,0,1}
\definecolor{greencomments}{rgb}{0,0.5,0}
\definecolor{redstrings}{rgb}{0.64,0.08,0.08}
\definecolor{xmlcomments}{rgb}{0.5,0.5,0.5}
\definecolor{types}{rgb}{0.17,0.57,0.68}

\lstdefinestyle{eclipse-style}{
	language=Java,
	showstringspaces=false,     
	basicstyle=\ttfamily\scriptsize,
	keywordstyle=\color{RoyalBlue}\ttfamily,
    stringstyle=\color{darkgreen}\ttfamily,
	commentstyle=\color{DarkPurple!60}\ttfamily,
	escapeinside={£}{£}, % latex scope within code      
	breaklines=true,
	breakatwhitespace=true,
	showspaces=false,
	showtabs=false,
	tabsize=2,
	morekeywords={length},
	numbers=none,
	numberstyle=\tiny\color{black},
	frame=none,
	aboveskip = 0em,
	belowskip = 0em
}
\lstset{
	style=eclipse-style
	% literate=  % Allow for German characters in lstlistings.
	% {Ö}{{\"O}}1
	% {Ä}{{\"A}}1
	% {Ü}{{\"U}}1
	% {ü}{{\"u}}1
	% {ä}{{\"a}}1
	% {ö}{{\"o}}1}
}

% Theorems \begin{mytheo}{title}{label}
\usepackage{tcolorbox}
\tcbuselibrary{theorems}
\newtcbtheorem[number within=section]{definiton}{Definition}%
{fonttitle=\bfseries}{def}
\newtcbtheorem[number within=section]{remember}{Merke}%
{fonttitle=\bfseries}{rem}
\newtcbtheorem[number within=section]{hint}{Hinweis}%
{fonttitle=\bfseries}{hnt}

% new section -> new page
% \let\stdsection\section
% \renewcommand\section{\clearpage\stdsection}

% Front page
\newcommand{\AUTHOR}{M. Panchaud}
\newcommand{\INSTITUTE}{Ostschweizer Fachhochschule}

%dotted rule
\usepackage{dashrule}
\usepackage{tikz}
\usetikzlibrary{decorations.markings}
\newcommand{\drule}[3][0]{
	\tikz[baseline]{\path[decoration={markings,
	mark=between positions 0 and 1 step 2*#3
	with {\node[fill, circle, minimum width=#3, inner sep=0pt, anchor=south west] {};}},postaction={decorate}]  (0,#1) -- ++(#2,0);}}


%no indentation
\setlength{\parindent}{0cm}

% DocInfo
\newcommand{\SUBJECT}{}
\newcommand{\TITLE}{Cheat Sheet Objektorientierte Programmierung 2}

\begin{document}

%do multicols
\begin{multicols*}{5}
    \setlength{\columnseprule}{0.4pt}
		\footnotesize

\section{Allgemeines}
	\subsection{?}

\section{Big O}

	\subsection{Beweis}
	\begin{itemize}
		\item Annahme: 2n + 10 ist O(n)
		\item $2n + 10 \leq cn$
		\item $10 \leq cn - 2n$
		\item $10 \leq (c - 2)n$
		\item $\frac{10}{c-2} \leq n$
		\item $=> c=3$ und $n0=10$
		\item Es gibt viele Belegungen für $c$ und $n0$
	\end{itemize}

\section{Verschiedene Algorithmen}
	\subsection{Eigenschaften}
	\renewcommand{\arraystretch}{1.1}
	\begin{tabular}{c | c | c}
		Algo & Worst & Avg \\
		\hline
		Binary Search & O(log n) & \\
		Insertion & O($n^2$) &  \\
		Selection & O($n^2$) &  \\
		Bubble & O($n^2$) &  \\
		Merge & O(n log n) &  \\
		Quick & O($n^2$) & O(n log n)
	\end{tabular}
	
	\subsection{Implementationen}
		\subsubsection{Binary Search}
			\begin{lstlisting}
public int searchBinary(int[] intArr, int start, int end, int searchElement) {
	if (start > end || intArr.length == 0) {
		return null;
	}

	int pivot = (start + end) / 2);
	if (pivot >= intArr.length){
		return null;
	}
	if (searchElement > intArr[pivot]) {
		searchBinary(intArr, pivot + 1, end, searchElement);
	} else if (searchElement < intArr[pivot] && start != pivot) {
		searchBinary(intArr, start, pivot - 1, searchElement);
	} else if(searchElement == intArr[pivot]) {
		return intArr[pivot]
	} else{
		return null;
	} }
			\end{lstlisting}
		
		\subsubsection{Insertion Sort}
		Für jedes neue Element wird geprüft, wo in der sortierten Liste es hingehört
			\begin{lstlisting}
void insertionSort(char[] data) {
	for (int k = 1; k < data.length; k++) {
		char cur = data[k];
		int j = k;
		while (j > 0 && data[j - 1] > cur) {
			data[j] = data[j - 1];
			j--;
		}
		data[j] = cur;
	} }
			\end{lstlisting}
		
		\subsubsection{Selection Sort}
		Kleinstes/grösstes Element suchen und an richtige Stelle verschieben
		
			\begin{lstlisting}
void selectionSort(int[] array) {
	int marker = array.length - 1;
	while (marker >= 0) {
		int max = 0;
		for (int i = 1; i <= marker; i++) {
			if (array[i] > array[max]) {
				max = i;
			}
		}
		swap(array, marker, max);
		marker--;
	} }
			\end{lstlisting}
		
		\subsubsection{Bubble Sort}
		Array von links nach rechts durchgehen, wenn Element grösser als rechter Nachbar, austausczhen
			\begin{lstlisting}
void bubbleSort(int[] array) {
	for (var n = array.length; n > 1; n--) {
		for (var i = 0; i < n - 1; i++) {
			if (array[i] > array[i + 1]) {
				swap(array, i, i + 1);
			}
		}
	} }
			\end{lstlisting}
		
		\subsubsection{Merge Sort}
		S auf Folgen S1 und S2 verteilen, rekursiv sortieren, S1 und S2 in sortierter Folge zusammenfassen
			\begin{lstlisting}
int[] mergeSort(int[] elem, int left, int right) {
	if (left == right)
	  	return new int[]{elem[left]};
	int middle = left + (right - left) / 2;
	int[] leftArray = mergeSort(elem, left, middle);
	int[] rightArray = mergeSort(elem, middle + 1, right);
	return merge(leftArray, rightArray);
}

int[] merge(int[] leftArray, int[] rightArray) {
	int leftLen = leftArray.length;
	int rightLen = rightArray.length;
	int[] target = new int[leftLen + rightLen];
	int targetPos = 0;
	int leftPos = 0;
	int rightPos = 0;// As long as both arrays contain elements...
	while (leftPos < leftLen && rightPos < rightLen) {
		// Which one is smaller?
		int leftValue = leftArray[leftPos];
		int rightValue = rightArray[rightPos];
		if (leftValue <= rightValue) {
			target[targetPos++] = leftValue;
			leftPos++;
		} else {
			target[targetPos++] = rightValue;
			rightPos++;
		}
	}
    // Copy the rest
	while (leftPos < leftLen) {
		target[targetPos++] = leftArray[leftPos++];
	}
	while (rightPos < rightLen) {
		target[targetPos++] = rightArray[rightPos++];
	}
	return target;
}
			\end{lstlisting}
		
		\subsubsection{Quick Sort}
		Pivot wählen, in zwei Arrays teilen, eines mit x < Pivot, eines mit x > Pivot, rekursiv auf Unterarrays Quicksort aufrufen
			\begin{lstlisting}
void quickSort(int[] arr, int low, int high) {
	if (low < high) {
		int pi = partition(arr, low, high);
		quickSort(arr, low, pi - 1);
		quickSort(arr, pi + 1, high);
	}
}
	
int partition(int[] arr, int low, int high) {
	int pivot = arr[high];
	int i = low - 1;
	for (int j = low; j <= high - 1; j++) {
		if (arr[j] < pivot) {
			i++;
			swap(arr, i, j);
		}
	}
	swap(arr, i + 1, high);
	return i + 1;
}
			\end{lstlisting}
		
\section{Bäume}
	\subsection{Höhe}
		\begin{lstlisting}
public int height(Position<E> p) {
	int h = 0;
	for (Position<E> c : children(p)) {
		h = Math.max(h, 1 + height(c));
	}
	return h; }
		\end{lstlisting}
	
	\subsection{Tiefe}
		\begin{lstlisting}
public int depth(Position<E> p) {
	if (isRoot(p)) {
		return 0;
	} else {
		return 1 + depth(parent(p));
	} }
		\end{lstlisting}
	\subsection{Traversierung}
		\subsubsection{Preorder W-L-R}
			\begin{lstlisting}
visit(v)
for each child w of v
  preOrder(w)
			\end{lstlisting}
		
		\subsubsection{Postorder L-R-W}
			\begin{lstlisting}
for each child w of v
  postOrder(w)
visit(v)
			\end{lstlisting}
		
		\subsubsection{Breadth-First}
		Alle Knoten der Tiefe t besuchen, bevor Knoten der Tiefe t+1 besucht werden
			\begin{lstlisting}
// Initialize queue Q containing root
while Q not empty
  v = Q.dequeue()
  visit(v)
  for each child w in children(v)
    Q.enqueue(w)
			\end{lstlisting}
		
		\subsubsection{Inorder L-W-R}
			\begin{lstlisting}
if hasLeft(v)
  inOrder(left(v))
visit(v)
if hasRight(v)
  inOrder(right(v))
			\end{lstlisting}

	\subsection{Preorder Iterator}
		\begin{lstlisting}
public T next() {
if (stack.empty()) {
	stack.push(root);
}
Node<T> node = stack.pop();
if (node.getRight() != null) {
	stack.push(node.getRight());
}
if (node.getLeft() != null) {
	stack.push(node.getLeft());
}
if (stack.empty()) {
	root = null;
}
return node.getValue(); }
		\end{lstlisting}


\section{Algorithmenparadigmen}
	\subsection{Backtracking}
		\subsubsection{Cookbook}
			\begin{itemize}
				\item Position markieren
				\item Rekursionsabbruch: return true
				\item Alle Optionen probieren
					\item => neues Feld festlegen
					\item => Überprüfen ob gültig \& nicht besucht
					\subitem => prüfen ob rekursiver Aufruf true? 
					\subitem Falls ja: return true
				\item Backtracking: Markierung vom Feld entfernen und return false
			\end{itemize}			
		
		\subsubsection{Beispiel Knights Tour}
			\begin{lstlisting}
private boolean isValid(int x, int y) {
	return x >= 0 && y >= 0 && x < N && y < N;
}

public boolean knightTour(int[][] visited, int x, int y, int pos) {
	// jetzige Position markieren
	visited[x][y] = pos;
	// alle Felder besucht => abbrechen (return true)
	if (pos >= N * N) {
		print(visited);
		return true;
	}
// Alle erlaubten Bewegungen überprüfen
	for (int k = 0; k < 8; k++) {
		int newX = x + row[k];
		int newY = y + col[k];
		// neue Position auf Feld und noch nicht besucht
		if (isValid(newX, newY) && visited[newX][newY] == 0) {
			// Wenn die Rekursion ein Weg gefunden hat
			if (knightTour(visited, newX, newY, pos + 1)) {
				return true;
			}
		}
	}
	
	// Markierung entfernen und backtracken
	visited[x][y] = 0;
	return false;
}
			\end{lstlisting}


\section{Beispiele}

	\subsubsection{Switch - Case}
	\begin{lstlisting}
int x = 5; 
switch(x) {
	case 2: 
		System.out.println("2"); 
		break;
	case 5: 
		System.out.println("5"); 
		break;
	default: 
		System.out.println("No match!"); 
}
	\end{lstlisting}

	\subsubsection{for loop}
	\begin{lstlisting}
	for(int i = 0; i < list.size(); i++) {}
	
	for(var el : list) {}
	
	for(var el : set) {}
	
	for(var el : map.entrySet()) {
		var k = el.getKey();
		var v = el.getValue();
	}
	\end{lstlisting}

	\subsubsection{Main Methode}
	\begin{lstlisting}
	public static void main(String[] args) {...}	
	\end{lstlisting}

	\subsubsection{Interface}
	\begin{lstlisting}
	public interface GraphicItem {
		Vector getLocation();
		void move(Vector delta);
		
		// A default implementation:
		default void moveToCenter() {
			var pos = getLocation();
			move(new Vector(-pos.getX(), -pos.getY()));
		}
	}
	\end{lstlisting}

	\columnbreak

	\subsubsection{Alle Wörter aus Buchstaben generieren}
		\begin{lstlisting}
Set<String> generateWords(Set<Character> input) { 
	Set<String> result = new HashSet<>();
	if (input.size() == 0) { 
		result.add(""); 
	}
	for (Character letter : input) {
		Set<Character> remainder = new HashSet<>(input); 
		remainder.remove(letter);
		for (String word : generateWords(remainder)) {
			result.add(word + letter);
		}
	}
	return result;
}
		\end{lstlisting}
	
	\subsubsection{Custom 'Linked List'}
	\begin{lstlisting}
public class Car {
	private Car next;
	// ...
}
public class Locomotive {
	private Car first;
	// ...
}
public class Train {
	private Locomotive locomotive;
	public Train() {
		locomotive = new Locomotive();
	}
	
	public void add(String name) {
		Car first = locomotive.getFirst();
		Car newCar = new Car(name);
		newCar.setNext(first);
		locomotive.setFirst(newCar);
	}
	
	public boolean insert(int position, String name) {
		if (position == 0) {
			add(name);
			return true;
		}
		Car leadingCar = getCar(position - 1);
		if (leadingCar == null) {
			return false;
		}
		Car followingCar = leadingCar.getNext();
		Car newCar = new Car(name);
		leadingCar.setNext(newCar);
		newCar.setNext(followingCar);
		return true;
	}
	
	public void reverse() {
		Car previousCar = null;
		Car currentCar = locomotive.getFirst();
		while (currentCar != null) {
			Car nextCar = currentCar.getNext();
			currentCar.setNext(previousCar);
			previousCar = currentCar;
			currentCar = nextCar;
		}
		locomotive.setFirst(previousCar);
	}
	
	public Car removeFirst() {
		Car firstCar = locomotive.getFirst();
		if (firstCar != null) {
			locomotive.setFirst(firstCar.getNext());
		}
		return firstCar;
	}
}
	\end{lstlisting}
	

\end{multicols*}

% \appendix

% List of figures
\listoffigures

% List of tables
\listoftables

\end{document}